\section{Introduzione}

Sviluppo software per diletto, studio e infine per lavoro, da ormai oltre dieci anni. Ho seguito l’evoluzione di ogni tecnica e metodologia, compreso l’avvento di nuovi \emph{framework} capaci di semplificare tutte le operazioni che fino a qualche anno fa mi impiegavano giorni o settimane.

Al primo anno di università, ho affrontato un grosso scoglio: educare la mia mente a ragionare in modo algoritmico. Non l’ho fatto da solo, ma insieme ad insegnanti e colleghi universitari. Sapevo a cosa andavo incontro durante lo studio di materie vicine a matematica o fisica, ma su tutto ciò che riguardava lo sviluppo del software ero completamente allo sbaraglio. Visto che tutto il percorso universitario era composto per almeno il 60\% da materie vicino all’\textsc{it}, posso confermare che è stato come una notte al museo durata cinque anni e due tesi.

Iniziando a lavorare, mi sono reso conto di quanto le persone risultino impreparate a dare un contributo di qualità allo sviluppo del codice sorgente. Tutti iniziamo con esperienze in \emph{stage} o tirocini come semplici programmatori; dopo un percorso universitario, è una cosa che si riesce a fare discretamente bene senza bisogno di tanta formazione aggiuntiva. Troppe volte mi è capitato di sviluppare in team composti da persone eterogenee e soprattutto su progetti partiti male e mantenuti peggio.

Credo molto nella condivisione del codice, ragion per cui sviluppo codice \emph{open-source} ogni volta che ne ho l’occasione. Perché, dunque, non condividere delle regole socialmente valide nell’\textsc{it}, in modo da aiutare la nascita di nuovi ottimi sviluppatori?

Questo articolo ha il \emph{solo} scopo di fornire un indirizzo. Sentitevi liberi di integrare e modificare.