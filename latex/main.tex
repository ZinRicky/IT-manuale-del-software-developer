%%%%%%%%%%%%%%%%%%%%%%%%%%%%%%%%%%%%%%%%%
% Journal Article
% LaTeX Template
% Version 1.4 (15/5/16)
%
% This template has been downloaded from:
% http://www.LaTeXTemplates.com
%
% Original author:
% Frits Wenneker (http://www.howtotex.com) with extensive modifications by
% Vel (vel@LaTeXTemplates.com)
%
% License:
% CC BY-NC-SA 3.0 (http://creativecommons.org/licenses/by-nc-sa/3.0/)
%
%%%%%%%%%%%%%%%%%%%%%%%%%%%%%%%%%%%%%%%%%

%----------------------------------------------------------------------------------------
%	PACKAGES AND OTHER DOCUMENT CONFIGURATIONS
%----------------------------------------------------------------------------------------

\documentclass[twoside,onecolumn]{article}

\usepackage{blindtext} % Package to generate dummy text throughout this template 

\usepackage[sc]{mathpazo} % Use the Palatino font
\usepackage[T1]{fontenc} % Use 8-bit encoding that has 256 glyphs
\linespread{1.05} % Line spacing - Palatino needs more space between lines
\usepackage{microtype} % Slightly tweak font spacing for aesthetics

\usepackage[italian]{babel} % Language hyphenation and typographical rules

\usepackage[hmarginratio=1:1,top=32mm,columnsep=20pt]{geometry} % Document margins
\usepackage[hang, small,labelfont=bf,up,textfont=it,up]{caption} % Custom captions under/above floats in tables or figures
\usepackage{booktabs} % Horizontal rules in tables

\usepackage{lettrine} % The lettrine is the first enlarged letter at the beginning of the text

\usepackage{quoting}
\quotingsetup{font=small}

\usepackage{enumitem} % Customized lists
\setlist[itemize]{noitemsep} % Make itemize lists more compact

\usepackage{abstract} % Allows abstract customization
\renewcommand{\abstractnamefont}{\normalfont\bfseries} % Set the "Abstract" text to bold

\usepackage{titlesec} % Allows customization of titles
\renewcommand\thesection{\arabic{section}}
\renewcommand\thesubsection{\thesection.\arabic{subsection}}
\titleformat{\section}[block]{\Large\scshape\centering}{\thesection.}{1em}{} % Change the look of the section titles
\titleformat{\subsection}[block]{\large}{\thesubsection.}{1em}{} % Change the look of the section titles
\setcounter{secnumdepth}{0}

\usepackage{fancyhdr} % Headers and footers
\pagestyle{fancy} % All pages have headers and footers
\fancyhead{} % Blank out the default header
\fancyfoot{} % Blank out the default footer
\fancyhead[C]{Manuale del Software Developer $\bullet$ 2020} % Custom header text
\fancyfoot[C]{\thepage} % Custom footer text

\usepackage{titling} % Customizing the title section

\usepackage[hidelinks]{hyperref} % For hyperlinks in the PDF
\usepackage{fontawesome}
%----------------------------------------------------------------------------------------
%	TITLE SECTION
%----------------------------------------------------------------------------------------

\setlength{\droptitle}{-4\baselineskip} % Move the title up

\pretitle{\begin{center}\Huge\bfseries} % Article title formatting
\posttitle{\end{center}} % Article title closing formatting
\title{Manuale del Software Developer  \large Semplici regole per migliorare le tue abilità da sviluppatore}
\author{
	\textsc{Carmelo La Gamba}\\[1ex]
	\normalsize
	\href{https://carmelolg.github.io/}{\faicon{github}}
	\href{https://t.me/carmelolg}{\faicon{paper-plane}}
}
\date{} % Leave empty to omit a date
\renewcommand{\maketitlehookd}{%
}

%----------------------------------------------------------------------------------------

\begin{document}

% Print the title
\maketitle
%----------------------------------------------------------------------------------------
%	ARTICLE CONTENTS
%----------------------------------------------------------------------------------------

%------------------------------------------------
\section{Introduzione}

Sviluppo software per diletto, studio e infine per lavoro, da ormai oltre dieci anni. Ho seguito l’evoluzione di ogni tecnica e metodologia, compreso l’avvento di nuovi \emph{framework} capaci di semplificare tutte le operazioni che fino a qualche anno fa mi impiegavano giorni o settimane.

Al primo anno di università, ho affrontato un grosso scoglio: educare la mia mente a ragionare in modo algoritmico. Non l’ho fatto da solo, ma insieme ad insegnanti e colleghi universitari. Sapevo a cosa andavo incontro durante lo studio di materie vicine a matematica o fisica, ma su tutto ciò che riguardava lo sviluppo del software ero completamente allo sbaraglio. Visto che tutto il percorso universitario era composto per almeno il 60\% da materie vicino all’\textsc{it}, posso confermare che è stato come una notte al museo durata cinque anni e due tesi.

Iniziando a lavorare, mi sono reso conto di quanto le persone risultino impreparate a dare un contributo di qualità allo sviluppo del codice sorgente. Tutti iniziamo con esperienze in \emph{stage} o tirocini come semplici programmatori; dopo un percorso universitario, è una cosa che si riesce a fare discretamente bene senza bisogno di tanta formazione aggiuntiva. Troppe volte mi è capitato di sviluppare in team composti da persone eterogenee e soprattutto su progetti partiti male e mantenuti peggio.

Credo molto nella condivisione del codice, ragion per cui sviluppo codice \emph{open-source} ogni volta che ne ho l’occasione. Perché, dunque, non condividere delle regole socialmente valide nell’\textsc{it}, in modo da aiutare la nascita di nuovi ottimi sviluppatori?

Questo articolo ha il \emph{solo} scopo di fornire un indirizzo. Sentitevi liberi di integrare e modificare.
\newpage
%------------------------------------------------

%------------------------------------------------
\section{Le dieci regole fondamentali}


%\subsubsection*{Se non ti piace sviluppare, non lo fare}

\begin{enumerate}

%1
\item \textbf{Se non ti piace sviluppare, non lo fare}

Durante la lettura di un libro, scritto da grandi \emph{software architect}, mi son trovato davanti la seguente frase: <<Pensiamo che non abbia senso sviluppare software se non si ha intenzione di farlo bene>>. Mi ha fatto molto riflettere perché a mio parere si può contestualizzare in ogni ambiente di lavoro. Inutile fare per forza qualcosa se non si riesce a contribuire nel migliore dei modi. Scrivere codice fatto bene è come un operazione chirurgica, tutto il team deve essere compatto e non ci si può permettere di lesionare parti del paziente. Se non è la tua strada proponiti per un altro dipartimento della tua azienda o in altri contesti. Revisionando la celebre frase <<Meglio un buon padre che un cattivo prete>>, con la stessa potenza mistica penso che sia \textbf{meglio fare qualcos'altro bene che sviluppare terribilmente}.

%2
\item \textbf{Di' <<no>> al tutto e subito}

Spesso, nei più svariati contesti, ci sono feature e correzioni da implementare a tempo zero (o addirittura per ieri). Ovviamente le situazioni di emergenza esistono e vanno trattate come tali. Se però l'emergenza diventa la quotidianità le opzioni sono due:
\begin{itemize}
\item Non siete nell'azienda giusta
\item A tutti i livelli di seniority c'è un problema di organizzazione da risolvere
\end{itemize}

In ogni caso il mio suggerimento è: \textbf{niente panico}. La prima cosa che lo sviluppatore medio pensa è: <<ora gli mollo un \emph{taccone} e in mezz'ora sono pronto per il deploy in produzione>>. Niente di più sbagliato. Agendo secondo questo principio del \emph{tutto e subito} stiamo solamente portando avanti la \textbf{teoria dei vetri rotti}. \\
La teoria dei vetri rotti sostiene, parafrasando, che un palazzo con una finestra rotta, potrebbe generare fenomeni di emulazione, portando qualcun altro a rompere un lampione o un idrante, dando così inizio a una spirale di degrado urbano e sociale. Ora il nostro palazzo è il codice sorgente e i passanti siamo noi programmatori. Il nostro compito è mantenere il palazzo in ottime condizioni e soprattutto evitare che qualcuno rompa le componenti. \\ \\
<<Si, bravo, belle parole, ma io ora come faccio? Devo consegnare fra mezz'ora>>. Nessun problema, l'allenamento è la miglior pratica: più ti allenerai a trovare la soluzione migliore e progettarla con altrettanta cura, più velocizzerai il modo di refattorizzare codice nel più breve tempo possibile. \'E una skill da non sottovalutare\dots e fa anche curriculum.

%3
\item \textbf{Sviluppa codice generico e riusabile}

Durante la progettazione e lo sviluppo di codice sorgente è naturale prendere come base la richiesta funzionale del cliente per mettere in piedi le prime classi, script, oggetti, funzioni etc.\\ Non è sbagliato ma si può incorrere in un errore molto comune cioè la presenza di codice inutilizzabile fuori da quel preciso contesto. Questa situazione porta a duplicazione, perdita di leggibilità e difficoltà nella manutenzione del codice. Tutto questo fortunatamente si può evitare scrivendo quanto più possibile \emph{codice generico}. \\ \\
Un esempio pratico è la scrittura di un file Excel, che consiste nella creazione dell'header, nella scrittura del contenuto e l'aggiunta finale di tutte le configurazioni necessarie.\\
Si può pensare di creare delle classi che generano file excel a prescindere dal contesto applicativo. Queste classi forniscono dei metodi per ricevere in input l'insieme dei valori da inserire nell'header e la lista dei valori da scrivere nel contenuto. Come assemblare il file, crearlo e restituirlo (ad esempio sotto forma di \emph{byte[]}) sarà in carico alle classi generiche. \\Se in altre parti del software sarà necessario creare un file Excel con altri requisiti funzionali, basterà richiamare la stessa classe e si eviterà parecchio codice duplicato. \\ \\
Quindi il mio suggerimento è: \textbf{quando progetti l'implementazione di una feature, accertati che le componenti siano generiche e riusabili}

%4
\item \textbf{Confrontati}

Il confronto in un team è il vero punto di forza. \\
Quando non sai come impostare il codice sorgente chiedi di fare pair-programming ad un collega o al Technical Leader di riferimento: le orchestre funzionano perché sono composte da tanti musicisti con strumenti diversi, non rischiare di fare un assolo di bombardino (non è bello e né orecchiabile, fidati).

%5
\item \textbf{Sviluppa codice \emph{ortogonale}} 

Questo consiglio è più che attuale ed è la base per le architetture di tipo SOA (Service-oriented architecture). \\
L'ideale è progettare e sviluppare componenti software ortogonali cioè completamente isolate. La modifica di ogni componente non deve interferire con il corretto funzionamento delle altre. Immaginate una navicella spaziale diretta verso Marte. A metà del viaggio uno dei pannelli solari smette di produrre energia, gli altri continuano a funzionare indisturbati evitando il collasso immediato di tutto il sistema vitale della navicella spaziale. Il software è come una navicella spaziale, ogni componente è a se stante, ognuna ha il suo scopo e deve avere una \textbf{singola responsabilità}.\\
Se si progetta software con componenti ortogonali si fa sempre la scelta giusta.

%6
\item \textbf{Ragiona e rifletti prima di scrivere codice}

Durante l'implementazione del codice sorgente, anche quando il tempo a disposizione non manca, tendiamo a non dedicare il tempo necessario per una progettazione adeguata. Magari in quel momento preferiamo optare per una risoluzione veloce e funzionante non badando al concetto di riusabilità e manutenibilità. In poche parole, stai per fare un taccone e nessuno ti potrà fermare. Per evitare di prendere numerosi insulti dai colleghi che malauguratamente si troveranno a dover mantenere quel codice, ti consiglio di porti la seguente domanda: \textbf{c'è un'alternativa più valida?} Esiste certamente dunque sviluppala, non vorrei mai che ti mandassero delle macumbe potentissime dove potresti rimanere offeso (semi-cit).

%7
\item \textbf{Sviluppa test adeguati}
Prima di sviluppare una funzionalità sarebbe meglio sviluppare i test automatici e di integrazione. Tutto ciò non è sempre possibile ma il mio consiglio è di provare sempre ad implementarli. Consapevole del fatto che tantissimo codice sorgente non è coperto da test unitari ci affidiamo alla possibilità che qualche anima di buon cuore possa implementarli anche successivamente: meglio tardi che mai (ma perché non essere proprio noi quell'anima di buon cuore?) 
\\
\textbf{E per i test utente?} Testa le evolutive per almeno 15 minuti.
Il numero è del tutto arbitrario, per definire una funzionalità <<ben testata>> si può scegliere il tempo necessario in base all'entità della \emph{feature}. Penso possa essere utile creare uno schema con tutte le casistiche da testare per ogni funzionalità. L'importante è essere certi di coprire i test utente adeguatamente.

%8
\item \textbf{Isolati durante lo sviluppo}

Ad un primo appuntamento è galateo spegnere il telefono per dedicare le giuste attenzioni al partner che si ha di fronte. Mentre si sviluppa codice non bisogna farsi distrarre da colleghi o agenti esterni, bisogna essere un'anima sola con il proprio computer. Il vostro partner è il computer. Procuratevi delle cuffie isolanti o se ne avete l'occasione andate in uffici o posti isolati per almeno due ore \dots e spegnete ogni notifica possibile e immaginabile.

%9
\item \textbf{Mantieni aggiornato il team}

Per portare avanti gli sviluppi ed organizzare il team in maniera semplice spesso si utilizzano strumenti come Board, Kanban o simili. Qualsiasi cosa utilizziate \textbf{aggiornatela}. Aggiornare il Team Leader e i colleghi è essenziale. E' come aggiungere dell'olio ad una catena di montaggio e anche una singola persona può inceppare la catena se non olia bene la sua componente.

%10
\item \textbf{Leggi libri tecnici}

Può aiutare tantissimo leggere libri tecnici, guide e manuali per implementare codice o architetture software adeguate. Per iniziare consiglio <<The Pragmatic Programmer>> di Hunt e Thomas.

\end{enumerate}

\newpage
%------------------------------------------------

%------------------------------------------------
\section{Le cinque buone maniere}

\begin{enumerate}

%1
\item \textbf{Rispetta sempre gli standard} \\
È sempre buona norma utilizzare standard per l’impostazione del codice. I \emph{design pattern} della Gang of Four sono un ottimo esempio. Potresti guadagnarti la stima e il rispetto dei colleghi, se produci software di qualità; viceversa, se ti lanci verso un \emph{hackathon} di \emph{spaghetti-code}, probabilmente finirai per essere il bersaglio durante il gioco a freccette nella sala relax aziendale.

%2
\item \textbf{Proteggi il \emph{branch} principale e segui un \emph{versioning} ben preciso} \\
Non lasciare che il \emph{branch} \texttt{master} sia accessibile da tutti: solo una persona (con una \emph{seniority} adeguata) deve avere l’onere di eseguire l’allineamento del \emph{branch} principale. Una volta scelto il \emph{pattern} di \emph{versioning}, seguilo sempre: evita eccezioni o di uscire dai binari. Prima di allineare i \emph{branch} condivisi (per esempio \texttt{develop}, \texttt{uat}, \texttt{collaudo}), testa il codice in maniera meticolosa.

%3
\item \textbf{Commenta e indenta il codice adeguatamente} \\
Ad inizio progetto, bisogna allineare le configurazioni di indentazione di ogni membro del team. Di solito gli \textsc{ide} ti consentono di sovrascrivere le configurazioni di \emph{default}. Sceglietene una e allineate tutti gli \textsc{ide} dei componenti del team. È anche un bel gesto commentare (senza scrivere un poema) il codice sorgente: i colleghi apprezzeranno. Creare una documentazione del codice sorgente, soprattutto per le componenti riusabili e generiche, è altrettanto un gesto apprezzato. Se segui questo consiglio, hai l’opportunità di diventare la \emph{rock star} del tuo \emph{open space}. Non mi farei sfuggire questa occasione.

%4
\item \textbf{Coinvolgi subito i colleghi interessati se sospetti un \emph{merge-conflict}} \\
Se stai sviluppando codice che impatta file sorgente attualmente in lavorazione (o già implementati) di altri componenti del team, coinvolgi subito i colleghi per evitare conflitti successivamente. Oltre ad essere una buona maniera, potresti evitare di perdere una numerosa quantità di codice già sviluppato.

%5
\item \textbf{Proponi \emph{pair-programming} in caso di \emph{refactoring}}

Se trovi del codice da refattorizzare e vuoi continuare il tuo percorso verso la santità, refattorizzalo. Sarebbe una grande azione anche coinvolgere l’autore del codice da refattorizzare: un buon sano \emph{pair-programming} migliorerà le \emph{skill} di programmazione di entrambi.

\end{enumerate}
\newpage
%------------------------------------------------


%------------------------------------------------
\section{Cinque consigli aggiuntivi}

\begin{enumerate}

%1
\item \textbf{Non sottovalutare mai i \emph{log}} \\
Proprio cinque minuti fa un mio collega mi ha scritto la seguente frase: «Ho un problema in produzione e dai \emph{log} non riesco a capire nulla. Puoi aiutarmi?». Pensate quanto costerà questa \emph{hotfix}: due persone per almeno due ore, per un problema che potrebbe essere risolto da una singola persona nel giro di poco tempo. Un buon sistema di \emph{logging}, seguendo degli standard (in alcuni casi imposti per legge), può salvarti da brutte situazioni più spesso di quanto immagini. \\
Usa dunque tutti i livelli di \emph{log} (i più usati sono \texttt{INFO}, \texttt{DEBUG}, \texttt{WARN}) adeguatamente e ringrazierai il giorno di averli inseriti nel codice.

%2
\item \textbf{Ricrea in locale un ambiente simile a quello in cui si effettua il \emph{deploy}} \\
Ricrea in locale l’ambiente di sviluppo o produzione. Ti eviterà di pronunciare la famigerata frase «A me in locale va», che è fastidiosissima e ormai non più accettata come giustificazione. Con strumenti come Docker è possibile ricreare l’ambiente adeguatamente senza troppi fronzoli.

%3
\item \textbf{Leggi il codice degli altri} \\
Per migliorare la propria abilità da programmatore si può leggere il codice di altri progetti. Molti grandi progetti di successo oggi sono \emph{open-source}, quindi vale la pena dare un’occhiata per scoprire nuovi modi di implementare codice e di applicare degli standard. È un po’ come leggere un buon libro: alla fine ti rimane qualcosa e al contempo hai avuto l’occasione di rivivere le nottate di qualche altro collega.

%4
\item \textbf{Proponi miglioramenti} \\
Proponi qualsiasi cosa, metti in campo le tue idee per migliorare lo sviluppo di un progetto. Si può proporre metodi o tecniche per minimizzare i tempi organizzativi (ad esempio su come diminuire i tempi di uno \emph{stand-up meeting} usando Scrum) oppure su come scrivere i commenti di un \emph{commit}. Tutto è utile, se proposto per migliorare la vita del team.

%5
\item \textbf{Partecipa ai meetup} \\
Fare parte di \emph{community} fuori dal contesto lavorativo può aiutare ad ingrandire il bagaglio culturale e tecnico di un professionista. Informati su gruppi attivi nella tua zona o nel tuo quartiere, partecipa ai \emph{meetup} organizzati e studia sempre nuove tecnologie che potrebbero interessarti.


\end{enumerate}
\newpage
%------------------------------------------------

%----------------------------------------------------------------------------------------
%	REFERENCE LIST
%----------------------------------------------------------------------------------------

\begin{thebibliography}{99} % Bibliography - this is intentionally simple in this template

\bibitem [Andrew Hunt, David Thomas]{}
Andrew Hunt e David Thomas (2018).
\newblock \href{https://www.amazon.it/dp/8850332548/ref=cm_sw_em_r_mt_dp_AvYCFbNVMFA1X}{Il Pragmatic Programmer. Guida per manovali del software che voglioni diventare maestri.}
\newblock {\em Apogeo}.
 
\end{thebibliography}

%----------------------------------------------------------------------------------------

\end{document}
