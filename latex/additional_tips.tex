\section{Cinque consigli aggiuntivi}

\begin{enumerate}

%1
\item \textbf{Non sottovalutare mai i log}.

Cinque minuti fa un collega mi ha scritto: «Ho un problema in produzione e dai log non riesco a capire nulla. Puoi aiutarmi?». Pensate quanto costerà quest’hotfix: due persone saranno impegnate per almeno due ore su un problema che potrebbe essere risolto da una singola persona nel giro di poco tempo. Un buon sistema di logging, seguendo degli standard (in alcuni casi imposti per legge), può salvarti da brutte situazioni più spesso di quanto immagini. 

Usa dunque tutti i livelli di log (i più usati sono \texttt{INFO}, \texttt{DEBUG} e \texttt{WARN}) adeguatamente e ringrazierai il giorno di averli inseriti nel codice.

%2
\item \textbf{Ricrea in locale un ambiente simile a quello in cui si effettua il deploy}.

Ricrea in locale l’ambiente di sviluppo o produzione. Ti eviterà di pronunciare la famigerata frase «a me in locale va», che è fastidiosissima e ormai non più accettata come giustificazione. Con strumenti come Docker è possibile ricreare l’ambiente adeguatamente senza troppi fronzoli.

%3
\item \textbf{Leggi il codice degli altri}.

Per migliorare la propria abilità da programmatore, si può leggere il codice di altri progetti. Molti grandi progetti di successo oggi sono open-source, quindi vale la pena dare un’occhiata per scoprire nuovi modi di implementare codice e di applicare degli standard. È un po’ come leggere un buon libro: alla fine ti rimane qualcosa e al contempo hai avuto l’occasione di rivivere le nottate di qualche altro collega.


%4
\item \textbf{Proponi miglioramenti}.

Proponi qualsiasi cosa, metti in campo le tue idee per migliorare lo sviluppo di un progetto. Si possono proporre tecniche per minimizzare i tempi organizzativi, ad esempio su come diminuire i tempi di uno stand-up meeting usando Scrum, oppure su come scrivere i commenti di un commit. Tutto è utile, se proposto per migliorare la vita del team.

%5
\item \textbf{Partecipa ai meetup}.

Fare parte di community fuori dal contesto lavorativo può aiutare ad ingrandire il bagaglio culturale e tecnico di un professionista. Informati su gruppi attivi nella tua zona o nel tuo quartiere, partecipa ai meetup organizzati e studia sempre nuove tecnologie che potrebbero interessarti.


\end{enumerate}